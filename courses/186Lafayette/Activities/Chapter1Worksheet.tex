\documentclass[12pt]{amsart}
\usepackage{amsmath,amsthm,amssymb,amsfonts,enumerate,fancyhdr}
\openup 5pt
%\author{Ann Clifton\\University of South Carolina}
\title{Math 186: Ch. 1 - What Statistics Is and Isn't Good For}
%\date{September 27, 2017}
\pdfpagewidth 8.5in
\pdfpageheight 11in
\usepackage[margin=1in]{geometry}

\renewcommand{\qedsymbol}{}

\begin{document}
\maketitle

\vspace{0.2in}
\makebox[\textwidth]{Name:\enspace\hrulefill}
\vspace{0.2in}




Here is Case 1.5 from pages 4--5 of the text (part of the assigned reading for today).\\ \\

{\narrower
News headlines are notorious for making one of the most common mistakes in the interpretation of statistical studies: jumping to unwarranted conclusions. A headline in {\it USA Today} read, ``Prayer can lower blood pressure'' (Davis, 1998). The story that followed continued the possible fallacy it began by stating, ``Attending religious services lowers blood pressure more than tuning into religious TV or radio, a new study says.'' The words ``attending religious services lowers blood pressure'' imply a direct cause-and-effect relationship. This is a strong statement, but it is not justified by the research project described in the article.

The article was based on an {\it observational study} conducted by the U.S. National Institutes of Health, which followed 2391 people aged 65 or older for 6 years (Figure 1.3). The article described the study's principal findings: `` People who attended a religious service once a week and prayed or studied the Bible once a day were 40\% less likely to have high blood pressure than those who don't go to church every week and prayed and studied the Bible less." (Davis, 1998) So the researchers observe a relationship, but it's a mistake to think that this justifies the conclusion that prayer actually causes lower blood pressure.

When groups are compared in an observational study, the groups usually differ in many important ways that may contribute to the observed relationship. In this example, people who attended church and prayed regularly may have been less likely than the others to smoke or drink alcohol. These could affect results because smoking and drinking are both believed to affect blood pressure. The regular church attendees may have had a better social network, a factor that could lead to reduced stress, which in turn could reduce blood pressure. People who were generally somewhat ill may not have been as willing or able to go out to church. We're sure you can think of other possibilities for {\it confounding variables} that may have contributed to the observed relationship between prayer and lower blood pressure.

{\bf Moral of the Story: } {\it Cause and effect conclusions cannot generally be made on the basis of an observational study.}

{\bf Definitions: } An {\bf observational study} is one in which participants are merely observed and measured. Comparisons base on observational studies are comparisons of naturally occurring groups. A {\bf variable} is a characteristic that differs from one individual to the next. It may be numerical, such as blood pressure, or it may be categorical, such as whether or not someone attends church regularly. A {\bf confounding variable} is a variable that is not the main concern of the study but may be partially responsible for the observed results. [For example, whether an individual was a smoker or not was a possible confounding variable in this study.]

}\vspace{.2in}
Now answer these questions:
\medskip
\newcounter{itemcounter}
\begin{list}
{\text{\arabic{itemcounter}.}}
{\usecounter{itemcounter}\leftmargin=1.4em}
\item With your partner, discuss what makes an observational study. Explain below why the study described here was labeled observational. You and your partner should agree on this explanation.
\vspace{2.5in}
\item The text explains that an observational study can't be used to demonstrate a cause and effect relationship, like whether prayer can lower blood pressure, because of {\it confounding variables}. Suggest a confounding variable that is not described in the text that might account for the association between attending religious services and lower blood pressure.
\newpage
One way to use statistics to establish a cause-and-effect relationship is via a {\it randomized experiment}. Case Studies 1.6 \& 1.7 in the text give examples. For example, to test whether a drug caused people's blood pressure to lower, a sample of people, called the subjects, would be divided at random into two groups, with members of one group receiving the drug and members of the other receiving a placebo. By placing subjects at random (for instance by a flip of a coin) into the groups, we make it highly unlikely that the two groups differ a great deal in the prevalence of smokers, and this is true for any other possible confounding variable, {\it including ones we don't think of beforehand}. As a result, we can be confident that if there is a difference in the drop of blood pressure after the two groups are treated, it will be due to the presence or absence of the drug.
\item Describe how you might design a randomized experiment to test the hypothesis that prayer can lower blood pressure.
\bigskip\bigskip\bigskip\vfill
\item Consider the outcome of your experiment. If it showed that there is no association between prayer and blood pressure, what would you conclude? Why?
\bigskip\bigskip\bigskip\vfill
\item What if the experiment showed an association between prayer and blood pressure; what would you conclude? Why?
\bigskip\bigskip\bigskip\vfill
\item Give a ``Moral to the Story'' from this thought experiment, like the ones at the end of the case studies in the text.
\bigskip\bigskip\bigskip\vfill
\item Did this thought experiment help your understanding of the limits of statistics?
\bigskip\bigskip\bigskip

\end{list}







\end{document}