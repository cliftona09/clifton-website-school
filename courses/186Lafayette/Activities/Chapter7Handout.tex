\documentclass[12pt]{amsart}
\usepackage{amsmath,amsthm,amssymb,amsfonts,enumerate,fancyhdr}
\openup 5pt
%\author{Ann Clifton\\University of South Carolina}
\title{Math 186: Ch. 7 - Probability}
%\date{September 27, 2017}
\pdfpagewidth 8.5in
\pdfpageheight 11in
\usepackage[margin=1in]{geometry}

\renewcommand{\qedsymbol}{}

\begin{document}
\maketitle

%\vspace{0.2in}
%\makebox[\textwidth]{Name:\enspace\hrulefill}
%\vspace{0.2in}


\newcounter{itemcounter}
\begin{list}
{\text{\arabic{itemcounter}.}}
{\usecounter{itemcounter}\leftmargin=1.4em}


\item[ ] {\bf Key Probability Definitions and Notation}\\

\subitem A {\bf simple event} is a unique possible outcome of a random circumstance.\\

\subitem The {\bf sample space} for a random circumstance is the collection of all simple events.\\

\subitem A {\bf compound event} is an event that includes two or more simple events.\\

\subitem An {\bf event} is any collection of one or more simple events in the sample space; events can be simple events or compound events.\\

\subitem Events are often written using capital letters $A$, $B$, $C$, and so on, and their probabilities are written as $P(A)$, $P(B)$, $P(C)$, and so on.\\

\subitem One event is the {\bf complement} of another event if the two events do not contain any of the same simple events {\it and} together they cover the entire sample space.  For an event $A$, the complement is denoted $A^c$.\\

\subitem Two events are {\bf mutually exclusive} if they do not contain any of the same simple events (outcomes); that is, the two events cannot occur simultaneously.  We also say two such events are {\bf disjoint}.\\

\subitem Two events are {\bf independent} of each other if knowing that one will occur (or has occurred) does not change the probability that the other occurs.\\

\subitem Two events are {\bf dependent} if knowing that one will occur (or has occurred) changes the probability that the other occurs.\\

\subitem The {\bf conditional probability of the event $B$, given that the event $A$ has occurred or will occur}, is the long-run relative frequency with which event $B$ occurs when circumstances are such that $A$ has occurred or will occur.  We denote this as $P(B\vert A)$.\\

\subitem Probabilities are always between 0 and 1 and the sum of the probabilities over all possible simple events is 1.

\newpage

\item[] {\bf Probability Rules}\\ \\

\item[] {\bf Simple Probability:} $$P(A)=\frac{\vert A\vert}{\vert S\vert}$$ where $\vert A\vert$ is the number of ways event $A$ can occur and $\vert S\vert$ is the total number of events in the sample space.

\vspace{.2in}

\item[] {\bf Complementary Events:} $$P(A^c)=1-P(A)$$

\vspace{.2in}

\item[] {\bf Addition and Multiplication Rules:}

$$\begin{array}{|r|ccccc}
\text{When Events Are:}&P(A {\text{ or }} B) \text{ is:}&\qquad&P(A {\text{ and }} B) \text{ is:}&\qquad&P(A\vert B) \text{ is:}\\ 
\hline
&\\
\text{Mutually Exclusive}& P(A)+P(B) && 0 && 0 \\ & \\ & \\
\text{Independent}& P(A)+P(B)-P(A)P(B) && P(A)P(B) && P(A) \\ & \\ & \\
\text{Any}& P(A)+P(B)-P(A{\text{ and }}B) && P(A)P(B\vert A) &&\dfrac{P(A{\text{ and }}B)}{P(B)}\\ & \\ & \\
\end{array}$$

\vspace{.2in}

\item[] {\bf Bayes' Rule:}

$$P(A\vert B)=\frac{P(A{\text{ and }}B)}{P(B\vert A)P(A)+P(B\vert A^c)P(A^c)}$$
























\end{list}
\end{document}