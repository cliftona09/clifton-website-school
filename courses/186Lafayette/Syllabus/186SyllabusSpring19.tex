\documentclass[10pt]{amsart}
\usepackage{amsmath,amsthm,amssymb,amsfonts,enumerate,hyperref,graphicx}
\openup 5pt
\author{Ann Clifton\\Lafayette College}
\title{Syllabus\\Math 186-05: Spring 2019}
%\date{August 24, 2017}
\pdfpagewidth 8.5in
\pdfpageheight 11in
\setlength{\parskip}{0pt}
\setlength{\parsep}{0pt}
\setlength{\headsep}{12pt}
\setlength{\topskip}{0pt}
\setlength{\topmargin}{0pt}
\setlength{\topsep}{0pt}
\setlength{\partopsep}{0pt}
\usepackage[margin=1in]{geometry}
\usepackage[compact]{titlesec}
\titlespacing{\section}{0pt}{*0}{*0}


\begin{document}
\maketitle

\section*{Welcome to Math 186}
Businesses, educators, politicians, physicians, and everyone in between now want to make decisions based on data.  You may have also heard of the buzzword Big Data, which has many people salivating at the promise of learning the key to untold fortune and fame from troves of data.  In all of these instances, though, especially now more than ever, it is paramount to have a firm grasp on the basic, fundamental statistical principles underlying much of the decision making, and to know the limits and caveats of using such methodology.  In this course, we will do just that - learn many of the basic principles underlying the majority of statistical tools and techniques, and try to understand the caveats associated with many studies and results drawn therefrom.  In short, our goal is to learn how to think critically and productively about data and statistical conclusions.

\section*{Contact Information}
\noindent
\begin{tabular}{p{1.4in}p{5in}}
  {\bf E-mail:} &\href{mailto:cliftona@lafayette.edu}{cliftona@lafayette.edu};\\ & Please use your University email for correspondence.\\
  & You are welcome to email me any time with questions and I will do my best to respond within 24 hours (48 on weekends).\\
  {\bf Office:} & 221 Pardee Hall.\\
  {\bf Office Hours:} & MWF 10:30am-12:00pm.\\
  & These are open office hours/help sessions - Feel free to drop by with questions!  Other times are available by appointment; email first, please.\\
  \end{tabular}

\section*{Course Information}
\noindent
\begin{tabular}{p{1.4in}p{5in}}
  {\bf Lectures:} & {\bf MW 12:45-2:00pm, 225 Hugel Science Center}\\
  			& Lectures will be given on Mondays and Wednesdays, and these will give an overview of the material that will be implemented in the Friday labs, and may also include problem-solving help.\\
  {\bf Labs:}	& {\bf F 12:45-2:00pm, 28 Pardee Hall}\\
  			& In the Friday meetings, you usually will work in groups to do activities that will help you master the reading and recent lecture material.\\
  {\bf SI Leader:} &  Cameron Zurmuhl, \href{mailto:zurmuhlc@lafayette.edu}{zurmuhlc@lafayette.edu}\\
 %{\bf Pre-Requisites:} & High School Trigonometry.\\
  {\bf Learning Outcomes:}
  & The main points of emphasis in the course include: methods for collecting and summarizing sample data,
   methods for evaluating the accuracy of estimates of unknown population values, and
   techniques for using sample data to make generalizations about larger populations.\\
 
 \end{tabular}
 
 
\section*{Course Information (cont'd)}
\noindent
\begin{tabular}{p{1.4in}p{5in}}
 
  {\bf Learning Outcomes (cont.):} & Upon successful completion of this course, students will be better able to:\\
  & 1) Understand the reasoning by which findings from sample data can be extended to larger, more general populations,\\
  & 2) Understand how to evaluate the results of research studies in which statistical analyses are done,\\
  & 3) Understand data collection issues and how to design a good statistical study,\\
  & 4) Analyze data using statistical software (R),\\
  & 5) Understand statistical examples and applications from a variety of fields.\\
  
  {\bf Software:} & This course uses {\bf R}, a free software environment for statistical computing and graphics.  It runs on Windows, MacOS, and a wide variety of UNIX platforms, including most flavors of Linux.  So, if you have a personal computer, R will probably run on it.  For labs, we will be using RStudio. You will be assigned a username and password that you must remember.\\
  {\bf Required Text:} & {\it Mind on Statistics,} 5/e, Cengage, by Utts and Heckard.\\
  {\bf Calculator:} & For exams, you will want a simple calculator (cell phones prohibited) that can compute square roots.
  \end{tabular}
  
%\section*{Course Information (cont'd)}
\noindent
\begin{tabular}{p{1.4in}p{5in}}
%  {\bf Required Text:} & {\it Calculus,} 7/e, Stewart, Cengage, 2016, ISBN 978-0-538-49781-7.\\
  {\bf Course Website:} & The course website will be maintained on Moodle (http://moodle.lafayette.edu) and will contain all course materials.  I use Moodle to communicate with students via email, so make sure you are enrolled in the proper sections and regularly check your Lafayette email, so you will receive important notices concerning this course. {\it You are expected to access the site regularly and you should access the website at nearly every Friday lab meeting.}  You are responsible for all material posted to the website.  
  
  
  %Additional practice problems are available on WebWork.  This online homework system allows for instant feedback on your work and gives you the opportunity to revise and correct your work an unlimited amount of times.  I highly recommend using the WebWork problems as a way to study and prepare for quizzes and exams.  In order to master the material, you will need to work many problems of varying types and levels of difficulty.
\end{tabular}

\section*{Coursework}
\noindent
\begin{tabular}{p{1.4in}p{5in}}
  {\bf Homework:} & Homework will be assigned regularly from the text and/or on WebWork.  The problems are chosen to highlight the core concepts from each section.  Mastery of these homework sets serves as a good indicator for exam performance.  As such, you should ensure that you fully understand the material on these homework sets; that is, upon completion of the homework set, you should be capable of completing similar problems without the aid of the text or any other tools not available during an exam.\\
  
  {\bf Attendance Quizzes:} & Attendance quizzes will be given once or twice per week (except when a test is scheduled) and will be based on the homework for the most recent sections covered.  Attendance quizzes will generally be 3-5 problems and should take between 10-15 minutes at the end of class.  Questions will be a combination of short answer questions, multiple choice questions, and applications.  Open materials and group work will be allowed. {\bf No make-up quizzes will be given.}\\
  & The lowest attendance quiz grade will be dropped at the end of the semester.\\
  
  {\bf Lab Quizzes:} & Lab quizzes will be given in most Friday labs, each with about 4 multiple-choice questions. {\bf No make-up quizzes will be given.}  The lowest lab quiz grade will be dropped at the end of the semester.\\
  
  {\bf Lab Completion/ Review Sessions:} & There will be weekly lab sessions during the Friday class time, held in Pardee 28.  (This is on the basement level of Pardee, near the Colton Chapel end of the hallway.) There we will learn to use R, and apply it to learning the concepts of statistics.  Our SI will help us in the lab weekly.\\
\end{tabular}
  \section*{Coursework}
\noindent
\begin{tabular}{p{1.4in}p{5in}}
    {\bf Exams:} & There will be two in class exams.  Tentative dates are given below.  The final exam is cumulative.
  {\bf{No make up exams will be given.}}  If you miss one exam, your final exam grade will replace the missing exam grade. This policy is intended only for exams missed due to illness, accidents, etc.  It does {\bf NOT} mean that your lowest exam grade will be dropped or replaced.  Any further missed exams will receive a zero. {\bf{Phones will not be allowed on exams.}}\\
  & {\bf Exam 1:} Monday, February 25\\
  & {\bf Exam 2:} Friday, April 12\\
  {\bf Final Exam:} & Final exam period: May 13-20, 2019.  The date for the cumulative final exam will not be set until a few weeks into class.  The date and time will be posted on Moodle and announced in class once it becomes available.  Until that time, {\bf do not} make any travel plans for the entire exam period.\\
\end{tabular}

\section*{Grading}
\noindent
\begin{tabular}{p{1.4in}p{5in}}
  {\bf Scale:} & Grades will be assigned on the following scale:\\
  & \begin{tabular}{llll}
      A: &93-100\% & C+: & 77-79.9\%\\
      A-: & 90-93\%,& C: & 70-76.9\%\\
      B+: & 87-89.9\%,& D: & 60-69.9\%\\
      B: & 83-86.9\%,& F: & $<$60\%\\
      B-: & 80-82.9\%, & &\\
      \end{tabular}\\
  {\bf Weights:} & Final grades will be calculated with the following weights:\\
  & \begin{tabular}{lr}
      Homework: & 15\%\\ 
      Attendance Quizzes: & 10\%\\
      Lab Quizzes: & 10\%\\
      Labs: & 10\%\\
      Exams: & 20\% each\\
      Final Exam: & 15\%\\ \hline
      Total: & 100\%\\
    \end{tabular}\\
\end{tabular}

\section*{Expectations}
\noindent
\begin{tabular}{p{1.4in}p{5in}}
  {\bf Academic Integrity:} & Students are expected to act in accordance with the {\it Lafayette College Student Code of Conduct}, 
  which can be found in the Student Handbook.  In particular,\\
  & To maintain the scholarly standards of the College and, equally important, the personal ethical standards of our students, it is essential that written assignments be a student?s own work, just as is expected in examinations and class participation. A student who commits academic dishonesty is subject to a range of penalties, including suspension or expulsion. Finally, the underlying principle is one of intellectual honesty. If a person is to have self-respect and the respect of others, all work must be his/her own.\\
 
  {\bf Attendance:} & Students are obligated to complete all assigned work promptly, to attend class regularly, and to participate in whatever class discussion may occur.  Sleeping, texting, studying for other classes, or any behavior that interferes with the ability of other students to learn is unacceptable.  If any of these become an issue, I will ask you to leave class for the day.\\
  & Your phone should be on silent and put away during class.  Texting and browsing social media during class is rude and disrespectful.  Please refrain from using your phone during class unless explicitly stated otherwise.  If you are caught with a cell phone on your desk, in your lap, or anywhere in view while taking a quiz or exam, it will be treated as a case of academic dishonesty.  Smart watches are also prohibited during quizzes and exams; please put them away during those times. \\
   \end{tabular}
  \section*{Expectations}
\noindent
\begin{tabular}{p{1.4in}p{5in}}
  {\bf Accommodations:} & In compliance with Lafayette College policy and equal access laws, I am available to discuss appropriate academic accommodations that you may require as a student with a disability.  Requests for academic accommodations need to be made during the first two weeks of the semester, except for unusual circumstances, so arrangements can be made.  Students must register with the Office of the Dean of Advising and Co-Curricular Programs for disability verification and for determination of reasonable academic accommodations.\\
  {\bf FAQ:} & {\bf How much time should I be spending on Math 186 each week?} A full-time job is considered 40 hours per week and a full-time student is considered to have a class schedule of 15 hours per week. If you subtract 15 hours of class time from the 40 hours, that leaves 25 hours of studying per week.  4/15 of 25 hours is roughly 7 hours of studying Math 161, outside of class time per week.\\
& If your last math class was several years ago or if your prerequisite math skills are weak, then you may need to spend considerably more time on this class in order to be successful!  If you are spending much more than 7 hours per week on this course, please come see me during office hours.\\
& Please note, just because you spend a lot of time studying does {\bf not} mean you ``deserve" the grade you want.  Effort does not equal mastery of material, which is what your final grade is based upon.
\end{tabular}

\section*{Campus Resources}
\noindent
\begin{tabular}{p{1.4in}p{5in}}
  %{\bf Additional Help:} & Free tutoring sessions held by the {\bf Calculus Calvary} are open to anyone on a walk-in basis.  All sessions are in Pardee 218.  Sessions will begin on February 4 with the following hours:\\
  %& $\cdot$ Mondays: 7-9 pm\\
  %& $\cdot$ Tuesdays: 4-6 pm\\
  %& $\cdot$ Wednesdays: 7-9 pm\\
  %& $\cdot$ Thursdays: 4-6 pm and 7-9 pm\\
  %& $\cdot$ Sundays: 4-6 pm\\
 {\bf Additional Help} & Try to form a {\bf study group} with your peers; making connections and learning with peers is an important aspect of the college experience.\\
  %& The {\bf Teaching Assistant} for this class, Keith, will have weekly office hours and is available to answer your questions.\\
  & Don't forget about me!  I am available during {\bf office hours} and by appointment to answer any question you may have.\\
  & The {\bf Academic Resource Hub/ATTIC}, located on the third floor of Scott Hall, provides academic services to enhance student success.  Resources available include:\\
  & $\cdot$ Tutoring and Supplemental Instruction,\\
  & $\cdot$ Academic Enrichment Resources,\\
  & $\cdot$ Accessibility Services,\\
  & $\cdot$ Services for Varsity Student Athletes.\\
  & For more information, see the website at \url{http://attic.lafayette.edu}.\\
  
  {\bf Disclaimer:} & I will try not to make changes to the syllabus during the course of the semester.  However, if changes are necessary, then they will be announced both in class and on Moodle and the revised syllabus will be posted on Moodle.\\
{\bf Compliance:}  & The student work in this course is in full compliance with the federal definition of a four credit hour course.  Please see the Lafayette College Compliance webpage (http://registrar.lafayette.edu/files/2012/07/Federal-Credit-Hour-Policy-Web-Statement.doc) for the full policy and practice statement.\\
\end{tabular}

%\newpage

%\begin{center}
%\includegraphics[scale=.8]{CourseOutline.png}
%\end{center}




\end{document}