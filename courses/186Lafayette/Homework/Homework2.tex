\documentclass{article}
\usepackage{amsmath, amssymb,graphicx}
\begin{document}
\begin{center}
  \Large Math 186: Homework 2\\
  \Large Due Monday, February 11
  \end{center}
%\vspace{.2in}
%\makebox[\textwidth]{Name:\enspace\hrulefill}
%\vspace{.2in}
%\newcounter{itemcounter}
%\begin{list}
%{\text{\arabic{itemcounter}.}}
%{\usecounter{itemcounter}\leftmargin=1.4em}

This assignment covers Sections 2.1-2.5 in the Utts and Heckard text and is designed to address the following learning objectives:\\

\begin{itemize}
\item To understand qualities of variables and data, and using that knowledge, appropriately summarize them (mostly using software like R) and interpret these summaries, both numerical and graphical
\subitem - To perform calculation of elementary statistical summaries for quantitative variables (like 5-number summaries) both by hand and with software
\subitem - To obtain frequency and two-way tables for one/two categorical variables using software
\item To recognize the salient features of quantitative variables (e.g. location, spread, shape, and outliers), know which graphical summaries are good at assessing each one, and know how these features interact (e.g. how shape influences measures of spread and location)
\end{itemize}

\hrulefill
\vspace{.2in}

\noindent
Provide solutions to the following problems from the text:\\ \\
Ch. 2 \#1, 6, 18, 30 [add to this question a part (e): summarize what you've learned from parts (b)-(d)], 38, 42, 62, 74


\end{document}