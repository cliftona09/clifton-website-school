\documentclass{beamer}
\usepackage{mystyle}
\graphicspath{{../imgs/}{../anims/}}
\pgfplotsset{compat=1.14}
\mode<presentation> {
  \usetheme{PaloAlto}
}

%%
\makeatletter
\setbeamertemplate{subsubsection in sidebar}{\vspace*{-\baselineskip}}
\setbeamertemplate{subsubsection in sidebar shaded}{\vspace*{-\baselineskip}}
\makeatother
%%

%%
\setbeamertemplate{theorems}[numbered]
%%

\definecolor{Garnet}{RGB}{150,0,33}
\usecolortheme[named=Garnet]{structure}

\logo{\includegraphics[width=1.5cm]{LafLogo2}}

%\setbeamercolor{title}{fg=red!60!black,bg=white!50!black}
%\usecolortheme{beaver}
%\usecolortheme{crane}
\usefonttheme{structuresmallcapsserif}
\usefonttheme[onlysmall]{structurebold}

\newtheorem{thm}{Theorem}
\newtheorem{lem}{Lemma}
\newtheorem{prop}{Proposition}
\theoremstyle{definition}
\newtheorem{defn}{Definition}
\newtheorem{rmk}{Remark}

%\newcommand*{\defeq}{\mathrel{\vcenter{\baselineskip0.5ex \lineskiplimit0pt
%                     \hbox{\scriptsize.}\hbox{\scriptsize.}}}%
%                     =}
\DeclarePairedDelimiter\ceil{\lceil}{\rceil}
\DeclarePairedDelimiter\floor{\lfloor}{\rfloor}

\input epsf

\title % (optional, use only with long paper titles)
    {Math 186}

\author[Clifton]
{Ann Clifton~\inst{1}}

\institute[USC]{
\inst{1}
Lafayette College}
%\inst{2}
%East Carolina University, Greenville, NC USA\\
%\inst{3}
%University of Johannesburg, Auckland Park, South Africa}

\date[January 17, 2017]
{Applied Statistics}

%\subject{Irredundant and Mixed Ramsey Numbers}
\setbeamercolor{alerted text}{fg=red!60!black}
\setbeamercolor{block title}{bg=white!50!black,fg=red!60!black}
\setbeamercolor{block title example}{bg=white!50!black,fg=red!60!black}
%\setbeamercolor{block title}{bg=white!50!black,fg=red!60!black}

\begin{document}

\begin{frame}
  \titlepage
\end{frame}

\begin{frame}
  \frametitle{Outline}
  \tableofcontents[pausesections]
\end{frame}


%\subfile{../content/2.2.tex} %Types of Variables -- 
%\subfile{../content/2.3.tex} %Summarizing Categorical Variables -- 
%\subfile{../content/2.4.tex} %Exploring Features of Quantitative Data with Pictures -- 
%\subfile{../content/2.5.tex} %Numerical Summaries of Quantitative Data -- 
%\subfile{../content/2.6.tex} %How to Handle Outliers
%\subfile{../content/2.7.tex} %Bell-Shaped Distributions and Standard Deviations
%\subfile{../content/3.1.tex} %Scatterplots
%\subfile{../content/3.2.tex} %Regression
%\subfile{../content/3.3.tex} %Measuring Strength and Direction with Correlation
%\subfile{../content/4.1.tex} %Displaying Relationships Between Categorical Variables
%\subfile{../content/Ch3Review.tex} %Voting Questions
\subfile{../content/8.1-8.4.tex} %Sections 8.1-8.5 -- 3/30/17


%\subfile{../content/4.2.tex} %Inflection Points -- 2/23/17
%\subfile{../content/4.3.tex} %Global Maxima/Minima 2/28/17 - 3/2/17
%\subfile{../content/2.5.tex} %Marginal Cost/Revenue -- 2/28/17 - 3/2/17
%\subfile{../content/4.4.tex} %Profit, Cost, and Revenue -- 2/28/17-3/2/17
%*************************EXAM 2*************************
%\subfile{../content/5.1.tex} %Distance and Accumulated Change -- 3/21/17 - 3/23/17
%\subfile{../content/5.2.tex} %The Definite Integral -- 3/28/13
%\subfile{../content/5.5.tex} %The Fundamental Theorem of Calculus -- 3/28/17

%\subfile{../content/6.3.tex} %Using the FTC -- 4/4/17
%\subfile{../content/5.3.tex} %Integration by Substitution -- 4/4/17
%\subfile{../content/6.6.tex} %Area Between Two Curves -- 4/6/17
%*************************EXAM 3*************************
\end{document}
