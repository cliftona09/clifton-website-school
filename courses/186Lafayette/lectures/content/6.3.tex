\documentclass[Lecture.tex]{subfiles}
\begin{document}
\section{6.3: Using the Fundamental Theorem to Compute Definite Integrals}

\begin{frame}{Example}
  Compute
  $$\int_1^3 2x\dx{x}.$$

  \begin{eqnarray*}
    \onslide<2->{\int_1^3 2x\dx{x} &=&}
    \onslide<3->{2 \int_1^3 x\dx{x}\\}
    \onslide<4->{&=&2 \left[\frac{1}{2}x^2\right]_1^3\\}
    \onslide<5->{&=& x^2\Big|_1^3\\}
    \onslide<6->{&=& 3^2 - 1^2\\}
    \onslide<7->{&=& 9 - 1\\}
    \onslide<8->{&=& 8.}
  \end{eqnarray*}
\end{frame}

\begin{frame}{Example}
  Compute
  $$\int_0^2 6x^2\dx{x}.$$

  \begin{eqnarray*}
    \onslide<2->{\int_0^2 6x^2\dx{x} &=&}
    \onslide<3->{6 \int_0^2 x^2\dx{x}\\}
    \onslide<4->{&=& \frac{6}{3} x^3\Big|_0^2\\}
    \onslide<5->{&=& 2\left(2^3 - 0^3\right)\\}
    \onslide<6->{&=& 2(8)\\}
    \onslide<7->{&=& 16.}
  \end{eqnarray*}
\end{frame}

\begin{frame}{Example}
  Compute
  $$\int_0^2 t^3\dx{t}.$$

  \begin{eqnarray*}
    \onslide<2->{\int_0^2 t^3\dx{t} &=&}
    \onslide<3->{\frac{1}{4}t^4\Big|_0^2\\}
    \onslide<4->{&=& \frac{1}{4}(16 - 0)\\}
    \onslide<5->{&=& 4.}
  \end{eqnarray*}
\end{frame}

\begin{frame}{Example}
  Compute
  $$\int_1^2 8x + 5\dx{x}.$$

  \begin{eqnarray*}
    \onslide<2->{\int_1^2 8x + 5\dx{x} &=&}
    \onslide<3->{8\int_1^2 x \dx{x} + 5\int_1^2\dx{x}\\}
    \onslide<4->{&=&\frac{8}{2}x^2\Big|_1^2 + 5x\Big|_1^2\\}
    \onslide<5->{&=& 4(4 - 1) + 5(2 - 1)\\}
    \onslide<6->{&=& 12 + 5\\}
    \onslide<7->{&=& 17.}
  \end{eqnarray*}
\end{frame}

\begin{frame}{Example}
  Compute
  $$\int_0^1 8e^{2t}\dx{t}.$$

  \begin{eqnarray*}
    \onslide<2->{\int_0^1 8e^{2t}\dx{t} &=&}
    \onslide<3->{8\int_0^1 e^{2t}\dx{t}\\}
    \onslide<4->{&=&\frac{8}{2}e^{2t}\Big|_0^1\\}
    \onslide<5->{&=&4(e^2 - e^0)\\}
    \onslide<6->{&=&4(e^2 - 1).}
  \end{eqnarray*}
\end{frame}


\end{document}
