\documentclass[Lecture.tex]{subfiles}
\begin{document}
\section{2.5: Numerical Summaries of Quantitative Variables}

\begin{frame}{Outliers}
Outliers have a bigger influence on the mean than on the median.\\ \pause
{\it Example:} Suppose we use the ages of death for a person's grandparents and great grandparents as an indicator of a ``typical" lifespan, and these ages are 76, 78, 80, 82, and 84.  Calculate the mean and median.\\ \pause
Mean:\\ \pause \begin{center}$\bar x=\dfrac{76+78+80+82+84}{5}=80.$\end{center}\pause
Median:\\ \pause \begin{center}$M=80$.\end{center}
\end{frame}
  
\begin{frame}{Outliers}
Now suppose that the youngest age was 46 instead of 76.\\ \pause
Median:\\  \begin{center}$M=80$\end{center}\pause
Mean:\\ \pause \begin{center}$\bar x=\dfrac{46+78+80+82+84}{5}=74$.\end{center}
\end{frame}
  
  
  
  
  
  
\end{document}
