\documentclass[Lecture.tex]{subfiles}
\begin{document}
\section{3.3: Measuring Strength and Direction with Correlation}




\begin{frame}{Correlation}
\begin{defn}
The statistical {\it correlation} between two quantitative variables is a number that indicates the strength and direction of a straight-line relationship.
\begin{itemize}
\item<1->
The strength of the relationship is determined by the closeness of the points to a straight line.
\item<2->
The direction is determined by whether one variable generally increases or generally decreases when the other variable increases.
\end{itemize}
\end{defn}\pause
\begin{rmk}
A statistical correlation describes only linear relationships. When the pattern is nonlinear, a correlation is not an appropriate way to measure the strength of the relationship.
\end{rmk}
\end{frame}

\begin{frame}{Interpreting the Correlation Coefficient}
\begin{itemize}
\item<1->
Correlation coefficients are always between -1 and 1.
\item<2->
The closer the correlation is to -1 or 1, the better a line represents the data.
\item<3->
If the correlation coefficient is exactly -1 or 1, then there is a perfect linear relationship and all data points fall on the same line.
\item<4->
A correlation of 0 indicates the best line through the data is horizontal.
\end{itemize}
\end{frame}

\begin{frame}{The Squared Correlation}
\begin{itemize}
\item<1->
A {\it squared correlation}, $r^2$, always has a value between 0 and 1 (sometimes expressed as a percent).
\item<2->
The phrase ``proportion of variation explained by $x$" is used in conjunction with $r^2$.
\end{itemize}
\end{frame}














\end{document}
