\documentclass[Lecture.tex]{subfiles}
\begin{document}
\section{3.2: Describing Linear Patterns with a Regression Line}

\begin{frame}{Regression Line}
  \begin{defn}
  A {\it regression line} is a straight line that describes how values of a quantitative response variable ($y$) are related, on average, to values of a quantitative explanatory variable ($x$).\pause  The equation for the line is called the {\it regression equation}.\pause A regression line is used for two purposes:
  \begin{itemize}
  \item<1->
  	To estimate the average value of $y$ at any specified value of $x$
  \item<2->
  	To predict the unknown value of $y$ for an individual, given that individual's $x$ value.
  \end{itemize}
 The term {\it simple linear regression} refers to methods used to analyze straight-line relationships.
 \end{defn}
\end{frame}

%Ball Bounce Back Activity

\begin{frame}{Regression Equation}
\begin{defn}
The {\it regression equation} for a regression line describes the relationship between $x$ and the {\it predicted values} of $y$, denoted $\hat y$: $$\hat y=b_0+b_1x$$
\end{defn}
\begin{rmk}
\begin{itemize}
\item<1->
 $b_0$ is the value of $\hat y$ when $x=0$
\item<2->
 $b_1$ is the {\it slope}
\end{itemize}
\end{rmk}
\end{frame}
  
%Interpolation/Extrapolation examples with classroom stats data

\begin{frame}{Statistical vs Deterministic}
\begin{defn}
\begin{itemize}
\item<1->
In a {\it deterministic relationship}, if we know the value of one variable, we can exactly determine the value of the other variable.  
\item<2->
In a {\it statistical relationship}, there is variation from the average pattern.  At best, we can provide an estimate for a value; the accuracy of that prediction depends on the natural variability from the overall pattern.
\end{itemize}
\end{defn}
\end{frame}

\begin{frame}{Summary}
\begin{itemize}
\item<1->
$\hat y$ estimates the average $y$ for a specific value of $x$.  It can also be used as a prediction of the value of $y$ for an individual with a specific value of $x$.
\item<2->
The slope of the line estimates the average or predicted increase in $y$ for each one-unit increase in $x$.
\item<3->
The intercept of the line is the value of $\hat y$ when $x=0$.  Note that interpreting the intercept in the context of statistical data makes sense only if $x=0$ is included in the range of observed $x$ values.
\end{itemize}
\end{frame}  
  
  
  
  
  
  
  
  


\end{document}
