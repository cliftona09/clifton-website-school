\documentclass[Lecture.tex]{subfiles}
\begin{document}
\section{2.2: Types of Variables}
\begin{frame}{Definition}
  \begin{defn}
    \begin{itemize}
    \item<1->
      A {\it categorical variable} is a variable for which the raw data are group or category names that don't necessarily have a logical ordering.
    \item<2->
      An {\it ordinal variable} is a categorical variable for which the categories have a logical ordering or ranking.
    \item<3->
      A {\it quantitative variable} is a variable for which the raw data are numerical measurements or counts collected from each individual.
    \item<4-> In a relationship between two variables, regardless of type, an {\it explanatory variable} is one that might partially explain the value of a {\it response variable} for an individual.
    \end{itemize}
  \end{defn}
\end{frame}


\end{document}
