\documentclass[Lecture.tex]{subfiles}
\begin{document}
\section{1.8: New Functions from Old}

\subsection{Function Composition}
\begin{frame}{Function Composition}
  \begin{defn}
    Given a function $f$ and a function $g$ such that the range of $f$ is contained in the domain of $g$ we can define the composition
    $$g \circ f(x) = g\left(f\left(x\right)\right).$$
  \end{defn}
  \onslide<2->{
  \begin{rmk}
    We require that the range of $f$ is contained in the domain of $g$ so that the composition makes sense.
    \onslide<3->{That is, we don't want $f(x)$ to be a point for which $g$ is undefined.}
  \end{rmk}
  }
\end{frame}

\begin{frame}{Example}
  Let
  \begin{itemize}
  \item<1->
    $f(x) = x + 1$, and
  \item<2->
    $g(x) = x^2$.
  \end{itemize}
  \onslide<3->{Both have domain and range $\R$, so we can compose in either order}.
  \onslide<4->{
    $$g \circ f(x) = g\left(f\left(x\right)\right) \onslide<5->{= g(x + 1)} \onslide<6->{= (x + 1)^2} \onslide<7->{= x^2 + 2x + 1.}$$}
  \onslide<8->{and
    $$f \circ g(x) = f\left(g\left(x\right)\right) \onslide<9->{= f(x^2)} \onslide<10->{= x^2 + 1.}$$}
\end{frame}


\begin{frame}{Example}
  Let 
  \begin{itemize}
    \item<1->
      $f(x) = \frac{1}{x}$, and
    \item<2->
      $g(x) = x - 1$.
  \end{itemize}
  \onslide<3->{The domain and range of $g$ are both $\R$}.
  \onslide<4->{The domain and range of $f$ are both
    $$\left\{x \in \R \;\middle\vert\; x \neq 0 \right\}.$$}
  \onslide<5->{If we restrict $g(x)$ to the domain 
    $$\left\{x \in \R \;\middle\vert\; x \neq 1\right\}$$
    then $g(x) \neq 0$.}
  \onslide<6->{Hence
    $$f \circ g(x) = \frac{1}{x - 1}.$$}
\end{frame}

\subsection{Scaling}

\begin{frame}{Vertical Scaling}
  \onslide<1->{Let $f(x)$ be a function and let $0 < a$ be a real number.}
  \onslide<2->{The graph of $af(x)$ is
    \begin{itemize}
      \item<3->a {\it vertical stretching} of the graph of $f(x)$ if $1 < a$
      \item<4->a {\it vertical shrinking} of the graph of $f(x)$ if $a < 1$.
    \end{itemize}
  }
\end{frame}

\begin{frame}{Examples}
  \begin{center}
    \includegraphics[scale=0.5]{scaling}
  \end{center}
\end{frame}
\subsection{Rigid Transformations}
\begin{frame}{Reflection}
  The graph of $-f(x)$ is a reflection of $f(x)$ across the $x$-axis.
  \begin{center}
    \includegraphics[scale=0.4]{reflections}
  \end{center}
\end{frame}

\begin{frame}{Vertical Shifting}
  Let $f(x)$ be a function.
  \onslide<2->{Let $0 < a$ be a real number.}
  \begin{itemize}
    \item<3->
      The graph of $f(x) + a$ is the graph of $f(x)$ shifted up $a$ units.
    \item
      <4->
      The graph of $f(x) - a$ is the graph of $f(x)$ shifted down $a$ units.
  \end{itemize}
\end{frame}

\begin{frame}{Examples}
  \begin{center}
    \includegraphics[scale=0.4]{vertShift}
  \end{center}
\end{frame}

\begin{frame}{Horizontal Shifting}
  Let $f(x)$ be a function.
  \onslide<2->{Let $0 < a$ be a real number.}
  \begin{itemize}
    \item<3->
      The graph of $f(x - a)$ is a horizontal shift of $f(x)$ by $a$ units to the right.
    \item<4->
      The graph of $f(x + a)$ is a horizontal shift of $f(x)$ by $a$ units to the left.
  \end{itemize}
\end{frame}

\begin{frame}{Examples}
  \begin{center}
    \includegraphics[scale=0.5]{horizShifts}
  \end{center}
\end{frame}
\end{document}
