\documentclass{beamer}
\usepackage[scaled=0.9]{helvet}
\usepackage{courier}
\usepackage{multicol}
\usepackage{xcolor}
\usepackage{color}
\usepackage{colortbl}
\usepackage{graphicx}
\usepackage{amsfonts}
\usepackage{amsmath, mathrsfs}
\usepackage{array}
\usepackage{calc}
\usepackage{float}
\usepackage{amssymb,amscd}
\usepackage{tabu}
\usepackage{hyperref}
%\hypersetup{pdfpagemode=FullScreen}
\def\noqed{\renewcommand{\qedsymbol}{}}

% \theoremstyle{plain}
% \newtheorem{proposition}{Proposition.}

\theoremstyle{definition}
\newtheorem{notation}{Notation}
\newtheorem{thm}{Theorem}
\newtheorem{cor}{Corollary}
\newtheorem{question}{Question}
\newtheorem{procedure}{Procedure}
\newtheorem{formula}{Formula}




%\usepackage{beamerthemesplit}
\usetheme{OxygenCCC}
\setbeamertemplate{background}{\includegraphics[width=\paperwidth,height=\paperheight]{background}}

\usepackage{amsfonts, amsmath, amsthm, amssymb}

\definecolor{AuburnOrange}{RGB}{221,85,12}
\definecolor{AuburnBlue}{RGB}{3,36,77}
\definecolor{AuburnSecondaryBlue}{RGB}{73,110,156}
\definecolor{AuburnSecondaryOrange}{RGB}{246,128,38}
\definecolor{AlabamaCrimson}{RGB}{163,38,65}
\definecolor{LSUpurple}{RGB}{70,29,124}
\definecolor{VanderbiltGold}{RGB}{207,181,59}



\title{\textcolor{yellow}{6.2: Binomial Probabilities}} % (optional, use only with long paper titles)
%\subtitle{\textcolor{yellow}{ }}

%\subtitle

% - Give the names in the same order as they appear in the paper.
% - Use the \inst{?} command only if the authors have different
%   affiliation.

\date{}

% \date[seminar] % (optional, should be abbreviation of conference name)

\newcommand{\ndiv}{\hspace{-4pt}\not|\hspace{2pt}}
%\centerline{\includegraphics[width=2 in]{square.jpg}}
\begin{document}

\frame{\titlepage}

 % \section[Outline]{}
 % \frame{\frametitle{Table of Contents} \tableofcontents}

\begin{frame}
\frametitle{Binomial Experiments}\pause
\begin{definition}
A \textbf{binomial experiment} is an experiment satisfying the following four conditions:
\begin{itemize}
\item There is a fixed number of trials, denoted $n$.\pause
\item The $n$ trials are independent and repeated under identical conditions.\pause
\item There are exactly two possible outcomes for each trial.  These outcomes can be considered \emph{success} and \emph{failure}.\pause
\item For each trial, the probability of success is the same.  We denote the probability of success by $p$ and the probability of failure by $q$.  Because each trial results in either success or failure, $p+q=1$. 
\end{itemize}\pause
The central problem of a binomial experiment is to find the probability of $r$ successes out of $n$ trials.
\end{definition}
\end{frame}

\begin{frame}
\frametitle{Binomial Experiments}\pause
\begin{example}
Determine if the following experiment  is a binomial experiment.  If it is not a binomial experiment, explain why.
\begin{itemize}
\item Selecting 20 university students and recording their class rank.
\end{itemize}
\end{example}\pause
This is not a binomial experiment because there are more than two outcomes for the variable.
\vspace*{3in}
\end{frame}

\begin{frame}
\frametitle{Binomial Experiments}
\begin{example}
Determine if the following experiment  is a binomial experiment.  If it is not a binomial experiment, explain why.
\begin{itemize}
\item Selecting 20 university students and recording whether they are on the Dean's list.
\end{itemize}
\end{example}\pause
This is a binomial experiment.
\vspace*{3in}
\end{frame}

\begin{frame}
\frametitle{Binomial Experiments}
\begin{example}
Determine if the following experiment  is a binomial experiment.  If it is not a binomial experiment, explain why.
\begin{itemize}
\item Drawing five cards from a standard deck of cards without replacement and recording whether they are red or black.
\end{itemize}
\end{example}\pause
This is not a binomial experiment because the probability of success will change with each draw.
\vspace*{3in}
\end{frame}

\begin{frame}
\frametitle{Binomial Experiments}\pause
\begin{example}
A survey from Teenage Research Unlimited found that $30 \%$ of
teenage consumers receive their spending money from part-time jobs.
We select 10 teenagers at random to determine the probability that exactly 4 of them will have part-time jobs.  Find the values $p,q,n,$ and $r$.
\end{example}\pause
\begin{itemize}
\item We will consider having a part-time job a success.\pause
\item Since $p$ is the probability of success, the example states that $p=0.3$.\pause
\item We can compute $q=1-p=0.7$.  Recall that $q$ is the probability of failure.\pause
\item We consider each selected teenager a trial.  So $n=10$.\pause
\item Since we want to consider the probability that exactly 4 of the selected teenagers will have a part-time job, $r=4$.
\end{itemize}
\end{frame}

\begin{frame}
\frametitle{Binomial Probability Distribution Formula}\pause
\begin{formula}
In a binomial experiment, the probability of $r$ successes out of $n$ trials is given by the formula
$$\Pr(r)=\frac{n!}{r!(n-r)!}p^r\cdot q^{n-r}=(C_{n,r})\cdot p^r\cdot q^{n-r}$$
where $p$ is the probability of success in each trial and $q$ is the probability of failure in each trial.
\end{formula}
\end{frame}

\begin{frame}
\frametitle{Binomial Probability Distribution Formula}\pause
\begin{example}
A survey from Teenage Research Unlimited found that $30 \%$ of
teenage consumers receive their spending money from part-time jobs.
If we select 10 teenagers at random, what is the probability that exactly 4 of them will have part-time jobs?
\end{example}
\begin{itemize}
\item In the previous example we found the following values
$$p=0.3\qquad\qquad\qquad q=0.7$$
$$n=10\qquad\qquad\qquad r=4$$\pause
\vspace*{-.1in}
\item Using the binomial probability distribution formula
$$\begin{array}{rcl}
\displaystyle \Pr(4)
&=&\displaystyle\frac{10!}{4!(10-4)!}(0.3)^4(0.7)^{10-4}\\ \\
&\approx&\displaystyle 0.2
\end{array}$$
\end{itemize}
\end{frame}

\begin{frame}
\frametitle{Binomial Probability Distribution Formula}\pause
\begin{example}
If a die is rolled 20 times, what is the probability that exactly half of the rolls will land on 3?
\end{example}\pause
\begin{itemize}
\item We begin by noticing that this is a binomial experiment.  Although there are six possible values on the die, we consider landing on a 3 a success and anything else a failure.\pause
\item Next we identify $n=20$, $r=10$, $p=1/6$ and $q=5/6$.\pause
\item Using the binomial probability distribution formula
$$\begin{array}{rcl}
\displaystyle \Pr(\text{Ten 3s})
&=&\displaystyle\frac{20!}{10!(20-10)!}\left(\frac{1}{6}\right)^{10}\cdot\left(\frac{5}{6}\right)^{20-10}\\ \\ \pause
&\approx&0.00049
\end{array}$$
\end{itemize}
\end{frame}

\begin{frame}
\frametitle{Mean and Standard Deviation}\pause
\begin{formula}
In a binomial experiment
\begin{itemize}
\item $\mu=np$
\item $\sigma=\sqrt{npq}.$\pause
\end{itemize}
The mean value $\mu$ can be thought of as the \textbf{expected number of successes} in the experiment.
\end{formula}\pause
\begin{example}
If we roll a single die 20 times, how many times can we expect 3 to roll?
\end{example}\pause
\begin{itemize}
\item Using the binomial experiment formula for $\mu$, we can expect the number of 3s rolled to be 
$$\mu=20\cdot\left(\frac{1}{6}\right)=3.\overline{3}$$
\end{itemize}
\end{frame}

\begin{frame}
\frametitle{Mean and Standard Deviation}
\begin{formula}
In a binomial experiment
\begin{itemize}
\item $\mu=np$
\item $\sigma=\sqrt{npq}.$
\end{itemize}
The mean value $\mu$ can be thought of as the \textbf{expected number of successes} in the experiment.
\end{formula}
\begin{example}
If we roll a single die 20 times, how many times can we expect 3 to roll?  Find the standard deviation for the number of 3s rolled.
\end{example}\pause
\begin{itemize}
\item Using the binomial experiment formula for $\sigma$, find the standard deviation for the number of 3s rolled to be
$$\sigma=\sqrt{20\cdot\left(\frac{1}{6}\right)\cdot\left(\frac{5}{6}\right)}=1.\overline{6}$$
\end{itemize}
\end{frame}


 
\end{document}